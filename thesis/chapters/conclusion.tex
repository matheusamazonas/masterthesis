\section{Discussion and Future Work}

\subsection{mTask}

The \gls{autohouse} application was successfully developed using \gls{mTask}, but some aspects still have to be analyzed. The tests performed during development were limited to five \gls{arduino} boards running a maximum of three simultaneous tasks for a maximum duration of one hour. It is unknown how \gls{mTask} performs running more than five simultaneous devices. Given that the number of devices in \acs{iot} applications can escalate quickly, it is important to assess how \gls{mTask} escalates in regard to the number of connected devices. Similarly, given that \acs{iot} applications often run continuously, it would be important to perform tests with tasks running for longer periods of time. Another characteristic of \acs{iot} solutions is lower power consumption~\cite{survey,survey2,survey3}. Although, no power consumption analysis of \gls{mTask} applications was performed. Assessing \gls{mTask}'s power consumption, scalability and behavior over long periods is suggested as future research. 

Ideally, the limitations of \gls{mTask} described in Section \ref{sec:limitations} should be eliminated. First, \acsp{sds} and tasks should be sent in a bundle. Thus, they would be removed altogether, eliminating dangling \acsp{sds}. Second, \gls{mTask} would avoid sending update messages to the exact same device which triggered the \acs{sds} update. Lastly, the \gls{mTask} library would provide callbacks to signal successful device connection and task acknowledgment. Solving these problems is also proposed as future research.

The Raspberry Pi\footnote{Raspberry Pi. Available at \url{https://www.raspberrypi.org}. Accessed on September 19th 2018} is a compact \acs{arm} computer often used in \acs{iot} projects both as a device and as a server. It would be interesting to see whether the Raspberry Pi is capable of running an \gls{mTask} application. Also, since the Pi hardware is Linux compatible, it might host an \gls{mTask} \acs{posix} client. Testing with the Raspberry Pi is suggested as future research.

The \gls{mTask} \acs{edsl} was created to bring \acs{iot} devices to the \gls{iTasks} environment. Its language is imperative and the system is not task-centred. Also, tasks cannot be combined as in \gls{iTasks}, limiting \gls{mTask}'s expressiveness. Naturally \gls{iTasks} and \gls{mTask} differ on some aspects, but ideally, the gap between them would be minimal, both in syntax and semantics. Current research has been performed to bridge this gap. A functional, task-centred version of \gls{mTask} has been proposed~\cite{micro}. Parallel and sequential task combinators were introduced, improving \gls{mTask}'s expressiveness. Due to time constraints, this version of \gls{mTask} could not be used in this research. Assessing the abilities of this new version of \gls{mTask} is proposed as future research.

\subsection{Autohouse}

Although \gls{autohouse} is not the research focus, it would be interesting to extend the application to push the boundaries of \gls{mTask}. 

Currently, devices have no information about what its peripherals represent. For example, a servo can be used to control curtains or to lock a door. The user might know what which peripheral represents when sending a new task to a device, but the migration algorithm does not. As a consequence, a task might migrate to a compatible device with a peripheral that controls a different object than the intended one. For example, a task that uses a servo to automatically close curtains might migrate to a device that uses its servo to lock a door. Ideally, the application user would attribute tags to its peripherals. The migration algorithm would take that information into consideration when migrating tasks.


\section{Conclusion}

The research reported in this documented tested \gls{mTask}'s ability to develop real-life \acs{iot} applications. The research question was tackled by example: the \gls{autohouse} application intended to assess \gls{mTask}'s capabilities. The application is a home automation system that allows users to dynamically manage automation tasks running on devices spread across different rooms. 

Limitations of \gls{mTask} surfaced during the development of \gls{autohouse}. Some limitations were overcome by changing the \gls{mTask} and CleanSerial libraries. Task variables were added to the language. Device disconnection recognition was implemented, allowing the application to automatically migrate tasks when a device is lost. A new view was added to the \ac{edsl} which generates minimum device requirements for a \gls{mTask} task. This view can be used to filter available devices based on whether they support a given task. Six new peripherals were added to the \gls{mTask} language and to the \gls{arduino} client. Peripheral code was restructured, easing the addition of new peripherals, increasing code maintainability and bringing a better separation of concerns between the language core constructs and peripheral constructs. Finally, the simulator for the interpreted \gls{mTask} was modified to support the setting of peripheral values and breakpoints, which improved testing and debugging considerably.

Other limitations could not be overcome during this research. \acsp{sds} are never removed from devices and live there indefinitely. There is an unwanted communication loop between devices and server whenever a device publishes an \acs{sds}. The \gls{mTask} library does not communicate task acknowledgment. Although these limitations were not overcome, they did not stop the development of \gls{autohouse}. 

The \gls{mTask} \acs{edsl} and library were successfully used to develop a real-life \acs{iot} application: the home automation system \gls{autohouse}. Some of the limitations unearthed during the development process were overcome and some remain. Finally, it is clear what the next steps to improve \gls{mTask} are.