Sometimes, interactions between the iTasks system and the real world could be automated. This is the case for tasks such as reading the room temperature or turning a LED on once a task is completed. Microcontrollers --- such as Arduino --- are perfect for this kind of task. They are appropriate for reading sensors (e.g., temperature, light) and for controlling actuators (e.g., motors, LEDs). In addition, due to the growth of \ac{iot}, they are becoming increasingly cheaper. Due to hardware limitations, microcontrollers can not run iTasks tasks. As an alternative, the mTask \ac{edsl} was created. This \ac{dsl} allows the programming of microcontroller in Clean using a TOP-like approach~\cite{clean,mtasks,mtasks2,martthesis}.

The mTask \ac{dsl} is a type safe, class-based, multi-view \ac{edsl} (\autoref{sec:class_based_edsl}). It currently has three views: iTasks simulation, C code generation and interpretable bytecode generation. 





\section{The Language}

\section{Interpreted mTask}

\section{Example}
