A \ac{gpl} is a computer language intended to be used in a wide range of domains. An example is the C++ programming language, which can be used in domains that vary from video games to web servers and systems programming. In contrast, a \acf{dsl} is a computer language that was designed to be used in a particular domain. Game Maker Language, \acs{html}, LaTeX and \acs{vhdl} are examples of \acp{dsl}. These languages, when compared to \acp{gpl}, offer a higher level of abstraction from their target domain.

A \ac{dsl} can be implemented using two different strategies: standalone and embedded. Each strategy has its advantages and disadvantages. In the former strategy, the language is built from the ground up, which consists of developing either a compiler or an interpreter. This strategy provides the language designer with a lot design freedom, but requires a cumbersome amount of work. In the latter strategy, the proposed \ac{dsl} is embedded in another language. Therefore, the \ac{dsl} inherits functionality from its host language. This inheritance often is desirable (e.g., a type checker) but sometimes might be undesirable (e.g., type coercion). This strategy frees its designer of building a new compiler.

The semantics of a \acf{edsl} are given by its views (also called backends). Common views are evaluation, pretty printing, compilation, optimization and verification. There are two main techniques for building an \ac{edsl}: deep and shallow embedding.

\section{Deep Embedding}

Building a deeply \ac{edsl} consists of representing language constructs as \acp{adt} in the host language. Views are functions that take the \ac{adt} as input and return another \ac{adt} that represents its semantics. An example of a simple deeply \ac{edsl} and its views (pretty printing and evaluation) can be seen on Listing \ref{deep1}.


\begin{lstlisting}[caption=A simple deeply \ac{edsl} and its views,captionpos=b,label=deep1]
:: MyDSL = I Int
    | B Bool
    | Add MyDSL MyDSL
    | Sub MyDSL MyDSL
    | And MyDSL MyDSL
    | Or MyDSL MyDSL
    | Var String
    
prettyPrint :: MyDSL -> [String]
eval :: MyDSL -> Int
\end{lstlisting}

The biggest advantage of deep embedding is that adding a view to the \ac{dsl} is easy: simply create a function that transforms the \ac{adt}. One disadvantage is that extending the \ac{dsl} might require a lot of work, since new code has to be created for every new construct in all the views. Another disadvantage is its lack of static type safety. As seen on Listing \ref{deep1}, \texttt{MyDSL} allows operations on mixed data types, as addition on booleans and disjunction on integers. In addition, variables are checked at runtime instead of compile time.

\acp{gadt} can be used to accomplish static type safety in deeply \acp{edsl}, but they do not enable compile time variable checks~\cite{gadts}. 

\section{Shallow Embedding}
Building a shallowly \ac{edsl} consists of representing the language constructs directly as its semantics. An example of a simple shallowly \ac{edsl} can be seen on Listing \ref{shallow1}. In this example, the \texttt{Sem} \ac{adt} contains both the evaluation (\texttt{a}) and the pretty printing (\texttt{[String]}) views.

\begin{lstlisting}[caption=A simple shallowly \ac{edsl},captionpos=b,label=shallow1]
:: Sem a = Sem (a, [String])

add :: (Sem a)    (Sem a)    -> Sem a    | + a
and :: (Sem Bool) (Sem Bool) -> Sem Bool
eq ::  (Sem a)    (Sem a)    -> Sem Bool | == a
var :: String -> Sem Int
\end{lstlisting}

One advantage of shallow over deep embedding is that adding a new language construct is easy, given that each construct is just a function. Another advantage is that overloading can be achieved with the use of class constraints. In addition, static type checking is obtained without \acp{gadt}. 

The biggest disadvantages of shallow embedding are based on the fact that all views are grouped in the same \ac{adt}. First, there is no separation of concerns. Second, there is computational waste when not all views are necessary. Finally, adding a new view in the semantics becomes increasingly burdensome. Another disadvantage of shallow embedding is that variables still remain unchecked during compilation. Finally, since the language constructs are functions, they can not be inspected as in deeply \acp{edsl}.

\subsection{Shallow Embedding with Type Classes}\label{sec:class_based_edsl}

Type constructor classes can be used to avoid computing all the views of a shallowly \ac{edsl}~\cite{tagless}. A type constructor class containing the language constructs is created and each of the \ac{dsl} views should provide an instance of the class. An example of a class-based shallowly \ac{edsl} can be seen on Listing \ref{shallow2}.

\begin{lstlisting}[caption=A simple class-based shallowly \ac{edsl},captionpos=b,label=shallow2]
:: Print a = P [String]
:: Eval a = E a

class myDSL v where
    lit :: t -> v t | toString t
    add :: (v t)    (v t)    -> v t    | + t
    and :: (v Bool) (v Bool) -> v Bool
    eq  :: (v t)    (v t)    -> v Bool | == t
    
instance myDSL Print where
    lit x = P [toString x]
    add (P x) (P y) = P (x ++ [" + ":y])
    and (P x) (P y) = P (x ++ [" && ":y])
    eq  (P x) (P y) = P (x ++ [" == ":y])
    
instance myDSL Eval where
    lit x = E x
    add (E x) (E y) = E (x + y)
    and (E x) (E y) = E (x && y)
    eq  (E x) (E y) = E (x == y)
\end{lstlisting}

Class-based shallow embedding solves most of the problems previously faced with deep and shallow embedding. The language is statically typed, the views are separated, adding a view is simple, extending the language with a new construct is easy and operators can be overloaded. 

Two problems remain. First, language constructs still can not be inspected. This is an inherent property of shallowly \acp{edsl} and unfortunately can not be eliminated. 

Second, variables are still not checked at compile time. We use functions to solve this. Given that function arguments are well-typed in the host language, we can use them to represent typed variables. A variable declaration is defined as a function where its argument represents the variable and the function body represents the remaining of the program. As a result, any usage of the variable in the function body is well typed. In order to avoid constructs that are not variables on the left-hand side of the attribution operator, the types \texttt{Var} and \texttt{Expr} were introduced~\cite{mtasks}. In some views, these types are phantom types~\cite{phantom}. 

An example of a class-based shallowly \ac{edsl} with compile time variable checks can be seen on Listing \ref{shallow3}. \texttt{Eval} is an example of a view with a phantom type: the type variable \texttt{b}.

\begin{lstlisting}[caption=A simple class-based shallowly \ac{edsl} with compile time variable checks,captionpos=b,label=shallow3]
:: Eval a b = E a
:: In a b = In infixl 0 a b
:: Var = Var
:: Expr = Expr

class expr v where
    lit :: t -> v t Expr
    (+.) infixl 6 :: (v t p) (v t q) ->  v t Expr | + t

class var v where
    var :: ((v t Var) -> In t (v a p)) -> v a p
    (=.) infixr 3 :: (v t Var) (v t p) -> v t Expr

instance expr Eval where
    lit x = E x
    (+.) (E a) (E b) = E (a + b)

instance var Eval where
    var _ = ...
    (=.) _ _ = ...

test1 :: Eval Int Expr
test1 = var \k = 4 In
    k  =. k +. lit 7

test2 :: Eval Int Expr
test2 = var \k = 4 In
    k  =. lit True     // Compilation   error
\end{lstlisting}

As seen in the example above, the variable \texttt{k} can be used after its declaration in a well typed manner. Function \texttt{test1} compiles successfully, as expected. Function \texttt{test2}, in the other hand, does not compile due to a type error. Therefore, compile time check of variables was achieved.

