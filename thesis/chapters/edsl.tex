A \ac{gpl} is a computer language that was designed with the purpose of developing software in a wide range of domains. An example is the C++ programming language, which can be used in domains that vary from systems programming to video games and web servers. In contrast, a \ac{dsl} is a computer language that was designed to be used in a particular domain. Game Maker Language, \acs{html}, LaTeX and \acs{vhdl} are examples of \acp{dsl}. These languages, when compared to \acp{gpl}, usually offer a higher level of abstraction from their target domain.

A \ac{dsl} can be implemented using two different strategies: standalone and embedded. Each strategy has its advantages and disadvantages. In the former strategy, the language is build from the ground up, which consists of developing either a compiler or an interpreter. This strategy provides the language designer with a lot design freedom, but requires a cumbersome amount of work. In the latter strategy, the proposed \ac{dsl} is embedded in another preexisting language. This strategy frees its designer of the task of building a new compiler, but limits the \ac{dsl} to the features of the language it is embedded in.

The semantics of a \ac{edsl} are given by its views (also called backends). Common views are evaluation, pretty printing, compilation, optimization and verification. There are two main techniques for building an \ac{edsl}: deep and shallow embedding.

\section{Deep Embedding}

Building a deep \ac{edsl} consists of representing language constructs as \acp{adt} in the host language. Views are functions that accept the \ac{adt} as input and return another \ac{adt} that represents its semantics. An example of a simple deep \ac{edsl} and its views (pretty printing and evaluation) can be seen on Listing \ref{deep1}.


\begin{lstlisting}[caption=A simple deep \ac{edsl} and its views,captionpos=b,label=deep1]
:: MyDSL = I Int
    | B Bool
    | Add MyDSL MyDSL
    | Sub MyDSL MyDSL
    | And MyDSL MyDSL
    | Or MyDSL MyDSL
    | Var String
    
prettyPrint :: MyDSL -> [String]
eval :: MyDSL -> Int
\end{lstlisting}

The biggest advantage of deep embedding is that adding a view to the \ac{dsl} is easy: simply create a function that transforms the \ac{adt}. One disadvantage of deep \acp{edsl} is that extending the \ac{dsl} might require a lot of work, since new code has to be created for every new construct in all the views. Another disadvantage is its lack of static type safety. As seen on Listing \ref{deep1}, \texttt{MyDSL} allows operations on mixed data types, as addition on booleans and disjunction on integers. In addition, variables are checked during runtime.

\acp{gadt} can be used to accomplish static type safety in deep \acp{edsl}, but it does not solve the type safety problem and it can not check variables during compilation. Also, \acp{gadt} are not available in Clean.

\section{Shallow Embedding}
Building a shallow \ac{edsl} consists of representing the language constructs directly as its semantics. An example of a simple shallow \ac{edsl} can be seen on Listing \ref{shallow1}.

\begin{lstlisting}[caption=A simple shallow \ac{edsl},captionpos=b,label=shallow1]
:: Sem a = Sem (a, [String])

add :: (Sem a) (Sem a) -> Sem a | + a
and :: (Sem Bool) (Sem Bool) -> Sem Bool
eq :: (Sem a)  (Sem a) -> Sem Bool | == a
\end{lstlisting}

One advantage of shallow over deep embedding is that adding a new language construct is easy, given that a each construct is just a function. Another advantage is that overloading can be achieved with the use of class constraints. In addition, static type checking was obtained without the usage of \acp{gadt}. 

The biggest disadvantages of shallow embedding are based on the fact that all views will always be computed regardless of its need. First, there is no separation of concerns. Second, there is computational waste when not all views are necessary. Finally, adding a new view in the semantics becomes increasingly burdensome. Another disadvantage of shallow embedding is that variables still remain unchecked during compilation. 

\subsection{Shallow Embedding with Type Classes}

Type classes can be used to avoid computing all the views of a shallow \ac{edsl}. The \ac{dsl} type class should contain the language constructs and each view should provide an instance of the class. An example of a simple shallow \ac{edsl} with classes can be seen on Listing \ref{shallow2}.

\begin{lstlisting}[caption=A simple shallow \ac{edsl} with classes,captionpos=b,label=shallow2]
:: Print a = P [String]
:: Eval a = E a

class myDSL v where
    add :: (v t) (v t) -> v t | + t
    andd :: (v Bool) (v Bool) -> v Bool
    eq  :: (v t) (v t) -> v Bool | == t
    
instance myDSL Print where
    add (P x) (P y) = P (x ++ [" + "] ++ y)
    andd (P x) (P y) = P (x ++ [" && "] ++ y)
    eq  (P x) (P y) = P (x ++ [" == "] ++ y)
    
instance myDSL Eval where
    add (E x) (E y) = E (x + y)
    andd (E x) (E y) = E (x && y)
    eq  (E x) (E y) = E (x == y)
\end{lstlisting}

Type class based shallow embedding solves most of the problems previously faced with deep and shallow embedding. The language is statically typed, the views are separated, adding a view is simple, extending the language with a new construct is easy and operators can be overloaded. The only remaining problem is that variables still are not checked. In order to enable compilation time variable check, one can take advantage of the fact that function parameters are statically typed in Clean. Listing \ref{shallow3} shows how to implement this technique in Clean.

\begin{lstlisting}[caption=A simple shallow \ac{edsl} with classes,captionpos=b,label=shallow3]
:: In a b = In infixl 0 a b

class var v where
	var :: ((v t) -> In t (v a)) -> v a
	
test = var \k = 4 In
	k  ==.  0
\end{lstlisting}

In the example above the variable \texttt{k} can be used anytime after its declaration, the behavior we wanted to accomplish. We leverage the Clean type system to enable compilation time variable checks with scoping.