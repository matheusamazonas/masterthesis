After the application was chosen, development began. This chapter describes the development process. Since the application was not the object under research, a quick overview of its development is given in Section \ref{sec:dev_overview}. During development, limitations of \gls{mTask} were found. Some were overcome and are described in Section \ref{sec:mtask_changes}. Other limitations remain and are described in Section \ref{sec:limitations}. Finally, Section \ref{sec:task_migration} describes how automatic task migration was accomplished without modifying \gls{mTask}.

\section{Development Overview}\label{sec:dev_overview}
\gls{autohouse} is an application developed using \gls{clean}, \gls{iTasks} and \gls{mTask}. Due to time constraints, it was thoroughly tested only on macOS 10.13 (High Sierra). It was tested on Linux (Ubuntu 16.10) on early stages of development. It was not tested on Windows at all because the CleanSerial library (which \gls{mTask} depends on) does not support it. \gls{autohouse}'s source code is available at its GitHub repository~\footnote{Autohouse on GitHub. Available at \url{https://github.com/matheusamazonas/autohouse}.}.

\subsection{Application Architecture}
\gls{autohouse} development started just like many \gls{iTasks} applications: defining \acsp{adt} and \acsp{sds}. The application has \acsp{adt} that model key concepts in the home automation domain: \texttt{House} (representing the smarthome), \texttt{Room} and \texttt{Unit} (representing a device). A \texttt{House} is a list of \texttt{Room}s and a \texttt{Room} is essentially a list of \texttt{Unit}s. 

All the application data lives in one \acs{sds} that represents the entire house. Other \acsp{sds} (e.g. for rooms and units) are derived from this main \acs{sds} using parametric lenses~\cite{parametric}. Rooms and units have unique ids that are used to locate them in the house \acs{sds}. 

The application contains three main tasks running in parallel: manage house, manage units and send new task. The first one lets the user create and edit rooms. The second allows the user to inspect, send tasks to and disconnect units. The last task lets the user pick a \gls{mTask} task to send to a device that is compatible with it.

Once the \gls{iTasks} foundation was created, the \gls{mTask} development could begin.

\subsection{Using the Simulator}
As expected, device features were first implemented using simulators (Section \ref{mtask_simulator}). These devices proved to be great for early stages of development. First, multiple simulators can be easily instantiated, which was particularly useful when physical devices were not available yet. Simulators allowed the development of peripheral tasks even before some peripherals were available for testing. Additionally, given that simulators are highly customizable, some features that depended on different device configurations were tested. Such a freedom of device setting choice does not exist when dealing with physical devices. 

Although the simulator was particularly helpful during early stages of development, it proved to be a great debug tool during later stages as well. Whenever a device behaved unexpectedly, the same environment was reproduced using a simulator. Then, using the debugging \acs{ui}, the device's state (i.e. memory, tasks, program counter, peripheral values) could be inspected. Using the simulator as a debug tool became standard practice during \gls{autohouse}'s development.

\subsection{Device Communication}
Once prototyping using the simulator was over, development moved to physical devices. But one problem had to be solved before deploying \gls{mTask} tasks on \gls{arduino}s: wireless device communication. Ideally, \gls{autohouse} would communicate with its devices wirelessly. But so far, only Serial connections over \acs{usb} were used to connect to \gls{arduino}s. Two wireless options were taken into consideration: Wi-Fi (through \acs{tcp}) and Bluetooth (through Serial).

The first option tested was Wi-Fi, specifically with the ESP8266. The ESP8266 is a system on a chip with a microcontroller, a full \acs{tcp}/\acs{ip} stack and Wi-Fi. It can be used either as a Wi-Fi module (giving other microcontrollers Wi-Fi capabilities) or as a standalone microcontroller. In \gls{autohouse}'s setting, it would be used to give the \gls{arduino} boards Wi-Fi capabilities. A microcontroller can use Hayes commands\footnote{Hayes command set - Wikipedia. Available at \url{https://en.wikipedia.org/wiki/Hayes_command_set}. Accessed on September 14th 2018.} to control the ESP8266 as a Wi-Fi module. Thus, a library was required to interface with it. A couple of open-source libraries were tested\footnote{WiFiEsp on GitHub. Available at \url{https://github.com/bportaluri/WiFiEsp/tree/master/src}. Accessed on September 14th 2018.}\textsuperscript{,}\footnote{ESP8266wifi on GitHub. Available at \url{https://github.com/ekstrand/ESP8266wifi}. Accessed on September 14th 2018.}, but they either did not offer some necessary features (e.g. retrieve the list of connected clients) or behaved unexpectedly (e.g. losing received messages). After some unsuccessful attempts to adapt the existing libraries, Wi-Fi was put aside and Bluetooth was considered as an alternative.

The Bluetooth module tested was the HC-05, a Bluetooth 2.0 \ac{spp} module. Since this module runs through Serial, it can be directly connected to the \gls{arduino} board's Serial pins (\texttt{TX} and \texttt{RX}). Incoming data is preprocessed by the HC-05 and send directly to the device's Serial port. Therefore, no client data processing is required to transmit data via Bluetooth. Additionally, no code changes were necessary to support Bluetooth connection. When compared to Wi-Fi, Bluetooth brought clear advantages --- i.e. it works out of the box and does not required additional code --- and therefore was chosen to be used as \gls{autohouse}'s wireless solution.

\subsection{Device Deployment}

Once the wireless communication was set up, development moved to physical devices. First, all peripherals were tested using a single \gls{arduino} board to ensure that they were compatible. Once all peripherals were tested using \gls{mTask} tasks, more devices were added to the \gls{autohouse} system. 

The test system contained five \gls{arduino} Uno compatible boards. Each board was equipped with a HC-05 Bluetooth module, two \acsp{led}, two push down buttons, a \acs{pir} and temperature, humidity, ultrasonic and analog light sensors. The boards had similar capabilities in order to test task migration (tasks should only migrate to devices that are compatible with it). Only one board was equipped with a \gls{servo}. This choice was also motivated by the task migration feature: if a device running a task that uses a \gls{servo} disconnects from the server, its task should not migrate because no other device has a \gls{servo}.

Once device deployment finished, \gls{autohouse}'s standard task list was created and tested. \gls{autohouse}'s version\footnote{\gls{autohouse} release 0.1.0 on GitHub. Available at \url{https://github.com/matheusamazonas/autohouse/releases/tag/0.1.0}.} 0.1.0 corresponds to the end of this development phase.

\section{Changes to mTask}\label{sec:mtask_changes}

Limitations of \gls{mTask} surfaced during the development of \gls{autohouse}. Some of these limitations were overcome by changing \gls{mTask}. These changes are described below.

\subsection{Variables}

Since \gls{mTask} is an imperative language, it would benefit from mutable data features. Although there are no \gls{mTask} constructs to represent variables, \acp{sds} might be used as updatable data containers. In such a setting, an \ac{sds} is created for each desired variable. This trick brings updatable data storage to \gls{mTask}, but it prompts two problems. 

First, there is no separation of concerns. Variables and \acp{sds} are by definition, different things. A variable is a \textit{local} updatable data storage in memory. An \ac{sds} is an abstraction layer over any kind of shared data, including data in memory. Using an \ac{sds} locally goes against what a \textit{shared} data source represents. Second, when \acp{sds} are sent to devices, they are not attached to a specific task. Also, on the current version of \gls{mTask}, there is no way to establish whether an \ac{sds} belongs to a given task. As a consequence, \acp{sds} are never deleted from devices. Variables, on the other hand, are always bond to a specific task and could be removed with their correspondent task altogether, saving space in the device's memory. Thus, \gls{mTask} could benefit from a language construct for variables. 

The \texttt{vari} class was created to fill this gap. It contains two functions: \texttt{vari} and \texttt{con}, representing variable and constant data storage respectively. Its definition can be seen in Listing \ref{vari_class}. From a language construct point of view, the \texttt{sds} and \texttt{vari} classes do not differ much. Both classes contain constructs that might be used as updatables and as expressions. But there are two differences between these classes. First, \texttt{vari} contains a construct for constant data: \texttt{con}. Second, \texttt{vari} functions expect a value of type \texttt{t} as its initial value (seen as the first argument of \texttt{In} in Listing \ref{vari_class}). The \texttt{sds} function expects a \texttt{(Shared t)} instead. The biggest difference between the \texttt{sds} and \texttt{vari} classes regards their behavior on the interpreted view of \gls{mTask}. Variables belong to a task and will live as long as the task lives. \acp{sds} are not bound to a task and will live in the device indefinitely. 


\begin{lstlisting}[caption=The \texttt{vari} class,captionpos=b,label=vari_class]
:: Vari  = Vari
instance isExpr Vari 
instance isUpd Vari 

class vari v where
  vari :: ((v t Vari) -> In t (Main (v c s))) -> (Main (v c s)) 
  con ::  ((v t Expr) -> In t (Main (v c s))) -> (Main (v c s))
\end{lstlisting}

Listing \ref{vari_example} displays an example of variables in \gls{mTask}: the task \texttt{blink}. This task blinks \texttt{LED1} based on the value of variable \texttt{v}. The variable \texttt{v} is created using the \texttt{vari} construct. Its value is updated using the \texttt{=.} infix operator, similarly to \acp{sds}. It can also be used as a boolean expression, as the condition to an \texttt{IF} construct.

\begin{lstlisting}[caption=Example of the usage of variables in mTask,captionpos=b,label=vari_example]
blink :: Main (v () Stmt) | program v
blink = vari \v = False In { main =
	IF (v) (
		ledOn (lit LED1)
	) (
		ledOff (lit LED1)
	) :.
	v =. Not v :. noOp
	}
\end{lstlisting}

The addition of variables to the language required changes on \gls{mTask}'s communication protocol (Section \ref{sec:mtask_com_prot}). When a task is sent to a device, its variables must be sent as well. Therefore, a \texttt{MTTask} message must include the variables used by the given task. Variables are modelled in the \texttt{BCVariable} record. A variable contains a unique (within a task) identifier and its initial value. The \texttt{BCVariable} record and the communication protocol change can be seen in Listing \ref{vari_message}.

\begin{lstlisting}[caption=Change in mTask's communication protocol to accommodate task variables,captionpos=b,label=vari_message]
:: BCVariable = { vid :: Int, vval :: BCValue }

:: MTaskMSGSend
	= MTTask Int MTaskInterval [BCVariable] String
	 ...
\end{lstlisting}

Additionally, the simulator and the client engine were modified to support task variables. When a task is received, its variables are stored. During task execution, variables are fetched and assigned similarly to \acp{sds}. When a task terminates, its variables are removed from the device.

\subsection{Peripheral Code}

The \gls{mTask} library already supported some of the peripherals \gls{autohouse} planned to use: \acsp{led}, analog and digital pins. Although, new peripherals (e.g. light, temperature and humidity sensors) were required by some of the default automation tasks. Following the natural development process of an \gls{mTask} application, these peripherals were first emulated using the simulator. As more peripherals were implemented, it was clear that the workflow required to add a new peripheral to the system could be improved.

Adding a new peripheral required changes on different parts of \gls{mTask}. An overview of the necessary changes can be seen below.

\begin{itemize}
    \item A new class that represents the peripheral is added to the language. 
    \item Depending on the peripheral, a new \ac{adt} is created to represent its values (e.g. \texttt{DigitalPin}).
    \item New bytecode instructions are created.
    \item Bytecode encodings are updated to support the new instruction and the possibly new \acs{adt}.
    \item The \texttt{MTaskDeviceSpec} record is modified to include a flag for the new peripheral.
    \item The simulator interpreter is updated to handle new bytecode instructions.
    \item The \gls{cpp} client is modified to handle the new peripheral.
\end{itemize}

The changes on the \gls{cpp} client code depended heavily on the type of peripheral being implemented. Changes on the \gls{clean} code though, were often similar. Previously, peripheral code was scattered around the \gls{mTask} library. Peripheral classes were inside the \texttt{Language} module along with possibly new \acsp{adt}. Instances of the peripheral classes for each \gls{mTask} view were in the respective view's module. The simulator interpreter contained peripheral-specific code. Bytecode encodings for basic types were mixed with encodings for peripheral data types. Overall, adding a new peripheral was particularly cumbersome and extremely error-prone. Finally, there was no separation of concerns whatsoever.

A new modular code architecture for peripherals was introduced to solve the problems described above. Each peripheral should be defined in its own module. Its type class, \acsp{adt}, bytecode encodings and view instances are defined in that same module. The simulator does not have any peripheral-specific code. Instead of explicit fields for each peripheral, the simulator state record (\texttt{SimState}) contains a list of \texttt{Peripheral}. This new data type is a wrapper around every \gls{mTask} peripheral. Its definition can be seen in Listing \ref{lis:peripheral}. 

\begin{lstlisting}[caption=The \texttt{Peripheral} class,captionpos=b,label=lis:peripheral]
:: Peripheral = E.e: Peripheral e & peripheral e

class peripheral e | iTask e where
	processInst :: BC e -> State SimState (e,Bool)
\end{lstlisting}

The \texttt{peripheral} class was created to enable the removal of peripheral-specific code from the simulator interpreter. Its only function, \texttt{processInst} defines how a peripheral should interpret bytecode instructions (\texttt{BC}). Naturally, a peripheral should only interpret instructions that are relevant to it. The simulator interpreter executes one instruction at a time. If an instruction belongs to \gls{mTask}'s core instruction set (excluding peripheral instructions), the interpreter executes it immediately. If the instruction does not belong to the core instruction set, it is assumed to be a peripheral instruction and it is presented to all simulator peripherals using the \texttt{processInst} function. Once a peripheral responds to an instruction (represented by the \texttt{Bool} on \texttt{processInst} returned value), the interpreter considers the instruction executed and stops looking for a peripheral to execute it. If no peripheral executes the instruction, an error ("instruction unknown") is thrown. 

The addition of new bytecode instructions remains outside of the peripheral module. Although technically it is possible to extend the bytecode data type (\texttt{BC}) in seperate modules, the amount of work necessary to do so outweighed the benefits it could bring. 

The development that followed the changes described above proved that the separation of concerns regarding peripheral code improved \gls{mTask}. Peripherals were added faster, with less code changes and less errors. Additionally, code maintainability increased substantially. Since peripheral code lays mostly in the same module, small changes can be performed faster and safer.

\subsection{New Peripherals}\label{sec:new_peri}

Previously, the interpreted \gls{mTask} supported the following peripherals: \acsp{led}, \acs{lcd} displays (for displaying numbers only), analog and digital pins. The standard \gls{autohouse} tasks required new peripheral support. The following peripherals were added to \gls{mTask}: DHT22 temperature and humidity sensor, HCSR04 ultrasonic sensor, digital light sensor, Grove analog light sensor (P) V1.1, \ac{pir} and \gls{servo}. 

The digital light sensor, the Grove analog light sensor and the \acs{pir} did not require an external library to be used. Their data pin is connected to board pins and their values can be read using \gls{arduino} standard functions. An additional library was required to interface with the \gls{servo}. The \gls{arduino} Servo library\footnote{Arduino Servo Library. Available at \url{https://www.arduino.cc/en/reference/servo}. Accessed on September 10th 2018.} was used. An additional library was also required to interface with the DHT22 sensor. Although there are libraries available out there, I decided to implement a small and simple one just for \gls{mTask}: DHTino\footnote{DHTino on GitHub. Available at \url{https://github.com/matheusamazonas/DHTino}. Accessed on September 10th 2018.}. This choice was motivated by the limited amount of flash memory (32 KB) on the \gls{arduino} Uno. Existing libraries support many different sensors and have many features that would not be used by \gls{mTask}. As a consequence, these libraries would take too much flash memory space. Guided by the same motivation, the Ultrino\footnote{Ultrino on GitHub. Available at \url{https://github.com/matheusamazonas/Ultrino}. Accessed on September 10th 2018.} library was created to interface with the HCSR04 ultrasonic sensor.


\subsection{Device Requirements}

Some tasks rely on certain peripherals to execute. For example, a task that regulates room temperature relies on a temperature sensor. Despite that, \gls{mTask} did not provide a mechanism to determine whether a task is compatible with a device. The \texttt{Requirements} view was created to bring this feature to \gls{mTask}. Its definition can be seen in Listing \ref{lis:requirements}. \texttt{Requirement} is a type constructor with two phantom type variables: \texttt{a} and \texttt{b}~\cite{phantom}. These type variables are required by \gls{mTask} type classes. \texttt{Requirement} is a wrapper around the device specification type \texttt{MTaskDeviceSpec}. 

Given a \gls{mTask} construct, this view will return the minimum device specification necessary to support that construct. This information can be used to determine whether a device matches the minimum specification for a task and therefore, if it is compatible with it. The \texttt{match} function (seen in Listing \ref{lis:requirements}) does exactly that. Given an \gls{mTask} program and a \texttt{Maybe MTaskDeviceSpec}, it yields whether the device and program are compatible.

\begin{lstlisting}[caption=The \texttt{Requirements} view,captionpos=b,label=lis:requirements]
:: Requirements a b = Req MTaskDeviceSpec

match :: (Main (Requirements a b)) (Maybe MTaskDeviceSpec) -> Bool

instance arith Requirements
instance UserLED Requirements 
\end{lstlisting}

Instances of \gls{mTask} classes (including peripheral classes) are defined for \texttt{Requirement}. Therefore, given a \gls{mTask} task, an application can filter the available devices based on whether they are compatible with it. The opposite is also possible: given a device, an application can filter tasks based on whether they are compatible with it.

\subsection{Device Disconnection}

By design, \gls{autohouse} should be robust regarding device disconnection (Section \ref{sec:app_analysis}). Ideally, the system would detect a device disconnection and migrate the device's tasks to another suitable device. There were two challenges to tackle in order to implement this feature. 

First, \gls{mTask} does not recognize device disconnection for all of the device types it supports. Simulators never get disconnected. \acs{tcp} devices throw an \gls{iTasks} error when a disconnection is identified. This error is not caught by mTask and propagates upwards. Serial devices kill the application when disconnected. The library used by \gls{mTask} to connect to Serial devices (CleanSerial\footnote{CleanSerial on GitLab. Available at \url{git@gitlab.science.ru.nl:mlubbers/CleanSerial.git}. Accessed on Semtember 8th 2018}) halts execution when a device is disconnected. 

In order to detect device disconnection, \gls{mTask} had to be modified. If the device communication fails, the \texttt{channelSync} task (Section \ref{sec:mtask_devices}) should throw an exception\footnote{An \gls{iTasks} Task yields either a value or an exception. The \gls{iTasks} standard library provides functions to create and handle exceptions.}. \acs{tcp} devices already throw an exception when communication fails and therefore require no change. Although simulators never disconnect from the system, simulating a disconnection would benefit testing. Hence, simulators were modified to support intentional disconnection. CleanSerial was modified to support device disconnection recognition. At this point, \gls{mTask} recognizes device disconnection, but does not communicate it to \gls{autohouse}. 

Ideally, \gls{mTask} would communicate device disconnection through an error handler that would be provided by the application. Thus, the application would decide what task to perform in case of a disconnection. As seen in Section \ref{sec:mtask_devices}, the \gls{mTask} library provides a single function to connect with a device: \texttt{withDevice}. This function is responsible (besides other tasks) to manage the connection to the device and therefore was the perfect place to insert an exception handler. An exception handler is a task that takes an error \texttt{String} as input. Listing \ref{with_device} displays the type signature of the original \texttt{withDevice} along with its new version, named \texttt{withDevice'} here. 

\begin{lstlisting}[caption=Change in mTask to support a device disconnection handler,captionpos=b,label=with_device]
withDevice  :: a (MTaskDevice -> Task b)                    -> Task b | channelSync a

withDevice' :: a (MTaskDevice -> Task b) (String -> Task ()) -> Task b | channelSync a
\end{lstlisting}

Consequently, \gls{mTask} recognizes and provides an exception handler for device disconnection. \gls{autohouse} uses this feature to detect unit disconnection and thus automatically migrate tasks from the disconnected device to a suitable one.

\subsection{Simulator Improvements}\label{sec:sim_improv}

The simulator (Section \ref{mtask_simulator}) proved to be an essential tool during the development of \gls{autohouse}. Although, it was clear that it could be improved to ease debugging and testing of the application. 

Sometimes, the developer might want to debug a task and inspect it closely. The simulator's manual mode is adequate for such usage, but it might be a bit cumbersome to use. Specially with large tasks, stepping over each program instruction becomes a rather tedious and inefficient process. With that in mind, the simulator was extended to support breakpoints on bytecode instructions. Tools to add and to step over breakpoints were added to the simulator \acs{ui}. When executing a task, the simulator goes through its bytecode instructions, checking if there are breakpoints on each instruction before executing it. If an instruction has a breakpoint, execution waits for user input (by clicking on "step over") to continue. At any point, the user is able to edit breakpoints. 

The ability to simulate peripheral values is crucial for program testing in \gls{mTask}. Tasks often rely on peripheral values and therefore can only be thoroughly tested if peripheral values can be simulated. Although, the simulator did not have such feature. The development of \gls{autohouse} showed how necessary this feature is for \gls{mTask} development. Hence, simulation of peripheral values was incorporated into the simulator. Values can be manually set via the simulator \acs{ui}, similarly to breakpoints. 

\section{Limitations of mTask}\label{sec:limitations}

\section{Task Migration}\label{sec:task_migration}

In case of device disconnection, the application should migrate the unit's tasks to another suitable one. The \gls{mTask} library offers means to connect and send task to devices, but not to manage them. The application using \gls{mTask} is responsible for device management. Therefore, automatic task migration was implemented entirely in \gls{autohouse} and required no changes to \gls{mTask} whatsoever.

First, the feature behavior was defined. Although automatic task migration might me helpful, not tall tasks should be automatically migrated. For example, some tasks that are suitable for a hallway unit (e.g. motion-activated light switch) might not be suitable for a bedroom unit. Therefore, different migration strategies were created. These strategies are represented in the \texttt{Migration} \acs{adt}, seen in Listing \ref{migration_types}. 

\begin{lstlisting}[caption=Task migration strategies of \gls{autohouse},captionpos=b,label=migration_types]
:: Migration = DoNotMigrate | SameRoom | AnyRoom
\end{lstlisting}

Tasks with \texttt{DoNotMigrate} migration strategy never migrate to another unit. Tasks with the \texttt{SameRoom} strategy migrate only to units within the same room of its original unit. Tasks with \texttt{AnyRoom} strategy migrate to any other unit in the smarthome. When a task is being migrated, other units are checked for compatibility (based on the task's strategy, following no particular order) and once a compatible unit is found, the task is sent to it. The \gls{autohouse} user is responsible for determining a task's migration strategy when sending it to a unit.

The application has to save enough task data to migrate a task in case of a device disconnection. The \gls{mTask}'s \texttt{liftmTask} function (Section \ref{sec:mtask_devices}) is used to send tasks to devices. Besides the device itself, this function takes a task interval and an \gls{mTask} task as input. Therefore, this is all the information the application needs to send a task to a device. \gls{autohouse}'s default tasks have a unique id that can be used to retrieve the task from the default task list. Thus, the application only needs to store the task index and its interval (\texttt{MTaskInterval}) in order to migrate it. 

Some \gls{autohouse} tasks require user-provided arguments to work. For example, a user must provide an initial target temperature for a thermostat task before sending it to a device. If such a task is being migrated, the application should be able to restore the task arguments as well. \gls{autohouse} stores task arguments along with the task index and interval. A task's index and arguments are stored in the \texttt{ProgramData} data type. All the information necessary to migrate an \gls{autohouse} program lays in the \texttt{ProgramInstance} data type. Both data types can be seen on Listing \ref{migration_data}.

\begin{lstlisting}[caption=Task migration data,captionpos=b,label=migration_data]
:: ProgramData :== (Int, [Dynamic])   // (index, arguments)

:: ProgramInstance :== (ProgramData, MTaskInterval, Migration)
\end{lstlisting}

A unit contains a list of \texttt{ProgramInstance}s. Each list element represents one task running on the unit and can be used to migrate its respective task to a new device. If a unit disconnects, the application goes through the unit's \texttt{ProgramInstance} list and migrates the tasks accordingly. Automatic task migration was tested using simulators and physical devices and it worked as expected.