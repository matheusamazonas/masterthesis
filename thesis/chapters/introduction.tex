
\section{Introduction}
\section{Research Question}
\section{Report Structure}

This report is structured as follows. Chapter \ref{chap:intro} contains the introduction, research question and report structure. Chapter \ref{chap:edsl} quickly introduces \acp{dsl} and presents different strategies to build \acp{edsl}. Chapter \ref{chap:top} introduces the concept of \ac{top} and the iTasks library. Chapter \ref{chap:mtask} presents the mTask \ac{edsl}, which was the research focus. Chapter \ref{chap:application} describes the real-life application developed during research. Chapter \ref{chap:dev} details the development process. Chapter \ref{chap:conclusion} concludes presenting insight about the development process, answering the research question and proposing future research. Chapter \ref{chap:related} lists related work.

The source code for the real-life application developed during research can be accessed at \url{https://github.com/matheusamazonas/autohouse}.

The source code of the modified version of mTask used during research can be accessed at \url{https://gitlab.science.ru.nl/mlubbers/mTask/tree/peripherals}.

The source code of this report can be accessed at \url{https://github.com/matheusamazonas/masterthesis}.

%The source code of the library to interface with the DHT22 temperature and humidity sensor can be accessed at \url{https://github.com/matheusamazonas/DHTino}.

%The source code of the library to interface with the HC-SR04 ultrasonic sensor can be accessed at \url{https://github.com/matheusamazonas/Ultrino}.