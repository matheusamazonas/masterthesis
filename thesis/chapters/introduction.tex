\section{Introduction}
% Structure: 
%   IoT ---
%   TOP and iTasks
% mTask
The \ac{iot} consists of a network of "things" (devices, computers, people, systems, etc) that interact with each other via the Internet. These components can exchange data, monitor and manage each other. \ac{iot} is a global, growing phenomenon. According to Gartner, there were 3.96 billion connected "things" in 2016 and by 2020, 12.863 billion devices are expected to be connected to the Internet~\cite{iot_numbers}. \ac{iot} is being used in a myriad of applications including house automation, fitness tracking, health care, warehouse monitoring, agriculture and industry manufacturing. \ac{iot} devices can be dedicated servers, personal computers, tablets, smartphones, smartwatches or compact devices operated by microcontrollers. Microcontrollers are small, cheap computers with limited resources and low power consumption commonly used to interface with the real world. They often gather data from sensors (movement, light, temperature, etc), act on actuators (motors, \acsp{led}, swtiches, etc) and communicate with other devices.

\ac{top} is a new programming paradigm used to develop online, collaborative applications. Its central concept, a \textit{Task}, can be used to model different types of work performed both by users and systems. \ac{top} provides a high level of abstraction, liberating the programmer from the burden of technical details, such as user interfaces. \ac{top} systems automatically generate as much assets as possible, turning it into a great tool for rapid prototyping. The iTasks~\cite{top} system implements \ac{top} in the functional programming language Clean~\cite{clean} as an \ac{edsl}. Given an iTasks program, the system automatically generates a web application that can be accessed via a web browser. Users can access this application to work on and inspect Tasks. The system has been proved useful in many fields, including incident response operations and navy vessels automation~\cite{incidone,navy}.


\section{Research Question}
\section{Report Structure}

This report is structured as follows. Chapter \ref{chap:intro} contains the introduction, research question and report structure. Chapter \ref{chap:edsl} quickly introduces \acp{dsl} and presents different strategies to build \acp{edsl}. Chapter \ref{chap:top} introduces the concept of \ac{top} and the iTasks library. Chapter \ref{chap:mtask} presents the mTask \ac{edsl}, which was the research focus. Chapter \ref{chap:application} describes the real-life application developed during research. Chapter \ref{chap:dev} details the development process. Chapter \ref{chap:conclusion} concludes presenting insight about the development process, answering the research question and proposing future research. Chapter \ref{chap:related} lists related work.

The source code for the real-life application developed during research can be accessed at \url{https://github.com/matheusamazonas/autohouse}.

The source code of the modified version of mTask used during research can be accessed at \url{https://gitlab.science.ru.nl/mlubbers/mTask/tree/peripherals}.

The source code of this report can be accessed at \url{https://github.com/matheusamazonas/masterthesis}.

%The source code of the library to interface with the DHT22 temperature and humidity sensor can be accessed at \url{https://github.com/matheusamazonas/DHTino}.

%The source code of the library to interface with the HC-SR04 ultrasonic sensor can be accessed at \url{https://github.com/matheusamazonas/Ultrino}.